\chapter{Preliminaries}\label{prelim}
\pagestyle{myheadings}\markboth{}{\small \ref{prelim} PRELIMINARIES}
\begin{spacing}{1.5}
\thispagestyle{empty}
\section{Problem Definition}
Complex networks are modeled as graphs whose vertices represent the nodes in the complex network and edges represent the relationship between two nodes. From here on, the complex network under consideration will be represented as a graph $G=(V, E)$, where $V$ represents the vertex set and $E$ the edge set.\\
\indent Communities are defined to be subsets of vertices such that the the number of edges between vertices in the same community is large and the number of edges between vertices in different communities is small. There are various possible definitions of a community and they are divided into mainly three classes: local, global and based on vertex similarity~\cite{Fortunato201075, Wasserman-Social-1994}. Let $S \subseteq V$ and $i \in S$. We define the internal degree and external degree of vertex $i$ with respect to $S$, denoted by $d_S^{in}(i)$ and $d_S^{out}(i)$, respectively, as follows:
\begin{align}
d_S^{in}(i) = |\{(i, j) \in E | j \in S\}|,\\
d_S^{out}(i) = |\{(i, j) \in E | j \notin S\}|.
\end{align}
The subset $S$ is a \emph{community in the weak sense}~\cite{Radicchi02032004} if:
\begin{align}
\displaystyle\sum_{i \in S} d_S^{in}(i) > \displaystyle\sum_{i \in S} d_S^{out}(i).
\end{align}
That is, the subset $S$ is a community in the weak sense if the sum of the internal degrees of all vertices in $S$ is greater than the sum of the external degrees of all vertices in $S$. The subset $S$ is a \emph{community in the strong sense}~\cite{Radicchi02032004}, if
\begin{align}
d_S^{in}(i) > d_S^{out}(i), \quad\forall i \in S.
\end{align}
That is, the subset $S$ is a community in the strong sense if for each vertex in $S$, its internal degree is greater than its external degree.\\
\indent The task of finding communities in graphs is usually modeled as an optimization problem. One of the most commonly used techniques is that of maximizing a quantity known as modularity~\cite{PhysRevE.69.026113}. It is a metric used to quantify the community structure found to determine how good of a structure we have obtained. The idea is that the density of edges connecting vertices in the same community should be higher than the expected density of edges between the same set of vertices if they were connected at random, but with the same degree sequence.\\
\indent Let $G=(V, E)$ be a graph on $n$ vertices. Let $C = \{C_1,\ldots,C_k\}$ be a set of communities in $G$. We define the modularity of $C$, denoted by $Q(C)$, as
\begin{align}
Q(C) = \displaystyle\sum_{i = 1}^k\left(\frac{e_i}{m} - \left(\frac{D_i}{2m}\right)^2\right),
\end{align}
where $e_i$ is the total number of edges inside the $i$th community and $D_i$ is the sum of the degrees of vertices in the $i$th community. So the first term represents the fraction of the total edges that are in the $i$th community and the second term represents the expected value of the fraction of edges if the vertices of the $i$th community were connected at random but with the same degree sequence.\\
\indent Modularity is a widely adopted metric to evaluate the community structure obtained on real world networks whose community structure is not known beforehand. High values of modularity indicate strong community structure. Using modularity as the objective function, the \emph{Community Detection Problem} (CDP) can now be formulated as:\\[5mm]
{\bfseries Community Detection Problem:}
\newline
\noindent \textbf{Input: }An undirected graph $G = (V, E)$.
\newline
\noindent \textbf{Output: }A set of communities $C = \{C_1, \ldots, C_k\}$ which represents the community structure of $G$ such that $\bigcup\limits_{1\leq i\leq k}C_i = V$ and $C_i\cap C_j = \varnothing$ for $1\leq i,j\leq k$, $i\neq j$ such that $Q(C)$ is maximum.\\[5mm]
\indent Brandes et al. showed that since maximizing modularity is an $\mathcal{NP}$-hard problem~\cite{10.1109/TKDE.2007.190689}, it is expected that the true maximum of modularity cannot be found in a reasonable amount of time even for small networks. Over the years, several heuristics have been developed for maximizing modularity. These are discussed in the next section.\\
\indent It is worth mentioning that while communities can also be hierarchical in nature, i.e., small communities can be nested within larger ones or overlapping, where each node may belong to multiple communities, in this work we only concentrate on finding disjoint communities. %The following section covers the previous work done regarding the task of community detection.
\section{Previous Work}
The seminal paper by Girvan and Newman~\cite{Girvan11062002} resulted in a lot of research into the area of community detection from various disciplines. As a result, these days there is a wide variety of community detection algorithms from fields like physics, computer science, statistics, etc. Covering all of them is beyond the scope of this work, for a more thorough review one can consult the comprehensive survey by Fortunato~\cite{Fortunato201075}.\\
\indent The methods for detecting communities can be broadly classified into hierarchical methods, modularity-based methods and other optimization methods involving statistics or dynamic processes on the graph.
\subsection{Hierarchical Methods}
Hierarchical community detection methods build a hierarchy of communities by either merging or splitting different communities based on a similarity criterion. The main idea here is to define the similarity criterion between vertices. For example, in data clustering where the points maybe plotted in 2D space, we can use Euclidean distance as a similarity measure. Hierarchical methods can be divided into two types based on the approach they take.\\
\indent Divisive hierarchical methods start from the complete graph, detect edges that connect different communities based on a certain metric such as edge betweenness~\cite{Girvan11062002}, and remove them. Edge betweenness of an edge is defined as the number of shortest paths between pairs of vertices that run through that edge. Edges connecting different communities have a high value of edge betweenness and by removing such edges iteratively, we can obtain the communities in the graph. Examples of divise hierarchical approaches can be found in~\cite{Girvan11062002, Radicchi02032004, PhysRevE.69.026113}.\\
\indent Agglomerative hierachical methods initially consider each node to be in its own community and then merge communities based on the criterion chosen, until the whole graph is obtained. Examples can be found in~\cite{newman03fast, blondel2008fuc, Clauset2004}. The criteria these algorithms use to merge communities is modularity.\\
\indent The disadvantage of hierarchical methods is that the result depends upon the similarity criteria used. Also, they return a hierarchy of communities whereas the network under consideration may not have any such hierarchical structure at all.
\subsection{Modularity-based Methods}
Modularity~\cite{PhysRevE.69.026113}, introduced in the previous section, is a metric for evaluating the community structure of a network. Values of modularity approaching 1 (which is the maximum), indicate strong community structure. In practice, modularity usually ranges from 0.3 to 0.7~\cite{PhysRevE.69.026113}.\\
\indent Under the assumption that high values of modularity indicate good community structure, the community structure corresponding to the maximum modularity for a given graph should be the best. This is the reasoning employed by modularity-based methods which try to optimize $Q(C)$ to find communities. These methods are also the most popular methods to be employed for community detection.\\% However, there are several different heuristics introduced over the years for maximizing modularity which perform very well.\\
\indent The first algorithm to maximize modularity was introduced in~\cite{newman03fast}. It is an agglomerative hierarchical approach where vertices are merged based on the maximum increase in modularity. Several other greedy techniques have been developed, some of these can be found in~\cite{blondel2008fuc,Clauset2004,Newman06062006,PhysRevE.74.016107}. Simulated annealing approaches to maximizing modularity are described in~\cite{Guimera04simulatedAnnealingNetworks, PhysRevE.71.046101}. Extremal optimization for maximizing modularity was used by Duch and Arenas~\cite{duch-2005-72}. Genetic algorithms have also been used for maximizing modularity~\cite{2008ppsnpizzuti,6045331, Pizzuti:2012:BDM:2245276.2245321}.\\
\subsection{Other Methods}
Various other techniques for community detection using methods based on statistical mechanics, information theory, random walks, etc., have been proposed.\\
\indent Reichardt and Bornholdt~\cite{PhysRevLett.93.218701} proposed a Potts model approach for community detection. In statistical mechanics, the Potts model is a model of interacting spins on a crystalline lattice. The community structure of the network is interpreted as the spin configuration that minimizes the energy of the spin glass with the spin states being the community indices~\cite{PhysRevLett.93.218701}. Another algorithm based on the Potts model approach is described in~\cite{PhysRevE.81.046114}.\\
\indent Random walks have also been used to detect communities. The motivation behind this is the idea that a random walker will spend a longer amount of time inside a community due to the high density of edges inside it. These methods are described in~\cite{PhysRevE.67.041908, ponslatapy05, vandongen00}. \\
\indent Information theoretic approaches use the idea of describing a graph by using less information than that encoded in its adjacency matrix. The aim is to compress the the amount of information required to describe the flow of information across the graph. The community structure can be used to represent the whole network in a more compact way. The best community structure is the one that maximizes compactness while minimizing information loss~\cite{1742-5468-2012-08-P08001}. Random walk is used as a proxy for information flow and the minimum description length (MDL) principle~\cite{Rissanen1978465} can be used to obtain a solution for compressing the information required. The most notable algorithm using this principle, referred to as Infomap, is described in~\cite{Rosvall29012008}.\\
\section{Ant Algorithms}
So far we have covered what the problem of community detection involves and the type of approaches that have been used to find the community structure in complex networks. To faciliate the understanding of our ant-based approach, a review of ant algorithms is given.\\ \indent Ant algorithms are a probabilistic technique for solving computational problems using artificial ants. The ants mimic the behavior of an ant colony in nature for foraging food. As they travel, ants lay down a trail of chemical called \emph{pheromone}, which evaporates over time. The higher the pheromone level on a path, the more likely it is to be chosen by the next ant that comes along.\\
\indent For example, consider a food source and two possible paths to reach it, one shorter than the other. Assume two ants set off on both paths simultaneously. The ant taking the shorter path will return earlier than the other one. Now this ant has covered the trip both ways while the other ant has not yet returned, so the concentration of pheromone on the shorter path will be more. As a result, the next ant will be more likely to choose the shorter path due to its higher level of pheromone. This leads to a further increase of pheromone on that path and eventually all ants will end up taking the shorter path.\\
\indent Thus, ants can be used for finding good paths within a graph. It is this basic idea that is used in ant algorithms for solving computational problems, but there are different variations. The first such approach, called Ant System (AS), was applied to the Traveling Salesman Problem by Marc Dorigo~\cite{dorigo1992}. Here, each ant is used to construct a tour and the pheromone level on all the edges in that tour is updated based on its length. An ant picks the next destination based on its distance and the pheromone level on that edge. A global update is applied everytime, which evaporates the pheromone on all edges.\\
\indent Since in AS each ant updates the pheromone globally, the run time can be quite high. Ant Colony System (ACS) was introduced to address this problem~\cite{585892}. In ACS, a fixed number of ants are positioned on different cities and each ant constructs a tour. Only the iteration best ant, the one with the shortest tour is used to update the pheromone. Ants also employ a local pheromone update in which the pheromone of an edge was reduced as an ant traversed it in order to encourage exploration.\\
\indent Another variation of AS, the Max-Min Ant System (MMAS), was introduced by St{\"u}tzle and Hoos~\cite{Stützle2000889}. The first change in this model is that the pheromone values are limited to the interval $[\tau_{min}, \tau_{max}]$. Secondly, the global update for each iteration is either done by the iteration best ant or the ant which has the best solution from the beginning. This is used to avoid early convergence of the algorithm. Additionally, the pheromone on each edge is initialized to $\tau_{max}$ so as to encourage exploration in the beginning of the algorithm. Apart from this, MMAS used the same structure of AS for edge selection and lack of local pheromone update. Both these variations were an improvement over the original AS.\\
\indent The techniques mentioned above fall into the category of ant colony optimization (ACO) methods. The approach used in our algorithm falls in to the category of ant-based optimization (ABO)~\cite{Bui:2009:PSM:1569901.1569903}. While in ACO, ants build complete solutions to the problem, in ABO ants are only used to identify good regions of the search space after which construction methods are used to build the final solution~\cite{5910378}. The ants only need local information as they traverse the graph. Choosing the next edge involves the pheromone level and some heurisitic information based on the rules specified for the ants.\\
\indent To the best of our knowledge, our algorithm is the first ABO method for detecting communities in complex networks. The next chapter describes in detail our ant-based algorithm for finding communities in complex networks.
\end{spacing}