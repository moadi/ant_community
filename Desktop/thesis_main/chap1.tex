\chapter{Introduction}\label{intro}
\pagestyle{myheadings}\markboth{}{\small \ref{intro} INTRODUCTION}
\begin{spacing}{1.5}
%\thispagestyle{empty}
Complex networks are extensively used to model various real-world systems such as social networks (Facebook and Twitter), technological networks (Internet and World Wide Web), biological networks (food webs, protein-protein interaction networks) etc. For example, in social networks the nodes represent people and two nodes are connected by an edge if they are friends with each other. In the World Wide Web,  nodes represent webpages and an edge represents a hyperlink from one webpage to another. In protein-protein interaction networks, nodes represent proteins and edges correspond to protein-protein interactions.\\
\indent Complex networks exhibit distinctive statistical properties. The first property is that the average distance between nodes in a complex network is short~\cite{milgram67smallworld}. This property is called the ``\emph{small world effect}''. The second property is that the degree distribution of the nodes follows a \emph{power-law}~\cite{Barabasi99emergenceScaling}. The degree of a node is the number of edges incident to it and the degree distribution of a network is the probability distribution of these degrees over the whole network. The power-law implies that this distribution varies as a power of the degree of a node. That is, the probability distribution function, $P(d)$, of nodes having degree $d$ can be written as $P(d) \approx d^{-\gamma}$, $d > 0$ and $\gamma > 0$, $\gamma$ is the exponent for the degree distribution. The third property, called \emph{network transitivity}, is that two nodes who are both neighbors of the same third node, have an increased probability of being neighbors of one another~\cite{Watts-Colective-1998}.\\
\indent Another property which appears to be common to such complex networks is that of a community structure. While the concept of a community is not strictly defined in the literature as it can vary with the application domain, one intuitive notion of a community is that it consists of a subset of nodes from the original network such that the number of edges between nodes in the same community is large and the number of edges connecting nodes in different communities is small. Communities in social networks may represent people who share similar interests or backgrounds. For technological networks such as the World Wide Web, communities may represent groups of pages which share a common topic. In biological networks such as protein-protein interaction networks, communities represent known functional modules or protein complexes.\\
\indent The problem of detecting communities is a very computationally intensive task~\cite{Fortunato201075} and as a result finding exact solutions will only work for small systems as the time it would take to analyze large systems would be too long. Therefore, in such cases it is common to use heuristic algorithms which do not return the exact solution but have an added advantage of lower time complexity making the analysis of larger systems feasible.\\
\indent Recently, the task of finding communities in complex networks has received enormous attention from researchers in different fields such as physics, statistics, computer science, etc. One of the most popular techniques used to detect communities is to model the task as an optimization problem, where the quantity being optimized is called \emph{modularity}~\cite{PhysRevE.69.026113}, which is used to quantify the community structure obtained by an algorithm. Maximizing modularity is one of the most popular techniques for community detection in complex networks. It is further examined in Chapter 2 and we present its drawbacks in Chapter 4. Other techniques for community detection involve using  dynamic processes running on a complex network such as random walks, or statistical approaches which employ principles based on information theory such as minimum description length (MDL) principle~\cite{Rissanen1978465} to find communities.\\
\indent Ant colony optimization (ACO) algorithms have been previously used to detect communities in complex networks~\cite{DBLP:journals/corr/abs-1303-4711, 5586496, Jin:2011:ACO:2022850.2022861}. In ACO, artificial ants are used in sequence to build a solution with later ants using information produced by previous ants. In ACO algorithms for finding communities, artificial ants are used to either optimize modularity or find small groups of well connected areas in the network which are used as seeds for building communities.\\
\indent In this thesis, we describe an ant-based optimization (ABO) approach~\cite{Bui:2009:PSM:1569901.1569903}, which is different from ACO, for finding communities in complex networks. We disperse a set of ants on the complex network who traverse the network based only on local information. The information produced by the ants is then used to build the first set of communities. Then, local optimization algorithms are employed to improve the solution quality before outputting the final set of communities.\\ 
\indent In ACO methods, each ant is used sequentially to construct a solution whereas in the ABO technique used here, a set of ants is used to identify good ``areas'' in the network, which are edges connecting nodes in the same community, so as to reduce the search space of the problem. Then, construction algorithms are used to build a solution to the problem.\\
\indent We have run our algorithm and six other community detection algorithms on a total of 136 problem instances out of which 128 are computer generated networks, with different degree distributions and community sizes, whose community structure is known. The remaining 8 are real world networks from different domains whose community structure is generally not known. Experimental results show that our algorithm is very competitive against the other approaches and in particular, it is very robust as it is able to uncover the community structure on networks with varying degree distributions and community sizes. \\
\indent The rest of this thesis is organized as follows. Chapter 2 provides more detailed information about the problem statement and covers the previous work done on the problem. Our ant-based algorithm is described in Chapter 3. Chapter 4 covers the metrics used to evaluate the community structure produced by an algorithm and Chapter 5 covers the performance of this algorithm on the problem instances and compares it to existing algorithms. The conclusion is given in Chapter 6.
\end{spacing}